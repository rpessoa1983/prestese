\documentclass{beamer}
\usepackage[utf8]{inputenc}
\usepackage[brazil]{babel}
\usepackage{graphicx}
\usepackage{amsmath}
\usepackage{subfig}
\usepackage{natbib}
\usepackage{bibentry}
\usepackage{booktabs} % Allows the use of \toprule,
\graphicspath{{img/}}
\usepackage{hyperref}

%%%%%%%%%%%%%%%%%%%%%%%%%%%%%%%%%%%%%%%%%%%%%%%%%%%%%%%%%%%%%%%%%%%%%%%%%%%%%%%%%%%%%%%%%%%%%%%%%%%%%%%%
\title{$q$-formalismo na dinâmica adaptativa por Extremum Seeking e na Termodinâmica de eletrólitos no contexto da digestão anaeróbia}

%\title{\large{Numerical Optimization Based on \textcolor{black}{Generalized} Extremum Seeking for \textcolor{black}{Fast} Methane Production by a Modified ADM1}}          %
%\author{Robson W. S. Pessoa$^{a}$, Felipe Mendes$^{a}$, Tiago Roux Oliveira$^{b}$,\\ 
%Karla Oliveira-Esquerre$^{a}$, Miroslav Krstic$^{c}$}
                                                                        %
                                                                                                       %
                                                                                                       %
                                                                                                       %
\titlegraphic{%
   \includegraphics[scale=.1]{img/brasao_ufba}\hspace*{3.25cm}~%
   \includegraphics[scale=.25]{img/logogamma}\hspace*{3.0cm}~%
   \includegraphics[scale=.2]{img/pei}\hspace*{0.75cm}%
   }                                                                                                       %
                                                                                                       %
%%%%%%%%%%%%%%%%%%%%%%%%%%%%%%%%%%%%%%%%%%%%%%%%%%%%%%%%%%%%%%%%%%%%%%%%%%%%%%%%%%%%%%%%%%%%%%%%%%%%%%%%


\begin{document}


\begin{frame}
\titlepage
\end{frame}

%sumario
%\begin{frame}
%\frametitle{Sumário} 
%\tableofcontents
%\end{frame}

\begin{frame}{Participantes}
	\begin{block}{Orientadores:}	 
	\begin{itemize}
    \item     Karla Patricia Santos Oliveira Rodríguez Esquerre  - Escola Politécnica - UFBA
    \item      Luciano Matos Queiroz - Escola Politécnica - UFBA
    \item      Tiago Roux de Oliveira, Faculdade de Engenharia - UERJ
    \end{itemize}
	\end{block}
	
	\begin{block}{Banca examinadora:}
	\begin{itemize}
  \item Prof. Dr. Enrique Lopez Droguett,  
	\item Prof. Dr. Ignacio Sebastián Gomez
	\item Prof. Dr. Diego Lima Medeiros
	\item Prof. Dr. Idelfonso Bessa dos Reis Nogueira
	\end{itemize}
	
	\end{block}

\end{frame}

\section{Documento da tese}

\begin{frame}
\begin{block}{Maximização da produção de metano}

\end{block}
\begin{block}{Termodinâmica de equilíbrio do metano}

\end{block}
\begin{block}{Teoria de Eletrólitos}

\end{block}

\end{frame}

\begin{frame}
\begin{block}{Tese}
\begin{figure}[ht] 
    \includegraphics[width=\linewidth]{diagramcap} 
    \end{figure}
\end{block}
\end{frame}

\section{Objetivo}


\begin{frame}{Objetivo}
\begin{block}{}
\begin{itemize}
\item Avaliar diferentes estruturas de extremum seeking baseada nas funções $q$-circulares.
\item Demonstrar a prova de convergência analítica para projeto de controles extremum seeking generalizado.
\item Analisar as implicações analíticas da generalização da função de perturbação sobre a velocidade convergência.
\item Investigar as consequências da aplicação do extremum seeking generalizado sobre o
efeito de mudanças paramétricas do modelo ADM1.
\end{itemize}
\end{block}
\end{frame}


\begin{frame}{Objetivo}
\begin{block}{}
\begin{itemize}
\item Comparar as estimativas de concentração de metano na fase líquida do modelo ADM1
sem correção do coeficiente de atividade iônica com modelos termodinâmicos com contribuições de interações de longo alcance e curto alcance.
 \end{itemize}
\end{block}
\end{frame}


\begin{frame}{Objetivo}
\begin{block}{}
\begin{itemize}
\item Propor equações de potencias eletrostáticos modificado pela $q$-exponencial para as geometrias esférica, plana e cilíndrica.
\item Comparar o desempenho do modelo de potencial de Debye-Hückel modificado pela $q$-exponencial para a estimativa de tensão superficial e coeficiente de atividade com modelos clássico e modificado.
 \end{itemize}
\end{block}
\end{frame}



\section{Energia e emissões GEE}

% O relatorio sintese aponta que a reducao de metano deve ocorrer 
% na industria especialmente em aguat/tratamento de água
\begin{frame}{AR6 Synthesis Report (SYR)}
\begin{columns}
\begin{column}{5cm}
\begin{block}{AR6}
\begin{figure}[ht] 
    \includegraphics[width=\linewidth]{AR6-SG3-IPCC} 
    \end{figure}
\end{block}
\end{column}
\begin{column}{5cm}
\begin{block}{Seção 4.5}
{\it "Rapid and deep reductions in GHG emissions require major energy system transitions (high confidence). Adaptation options can help reduce climate-related risks to the energy system (very high confidence)."}
\end{block}
\end{column}
\end{columns}
\end{frame}



\begin{frame}{Emissões Fugitivas}
\begin{columns}
\begin{column}{5cm}
\begin{block}{}
\begin{figure}[ht] 
    \includegraphics[width=\linewidth]{cad_tec_eng_san_abes_2021} 
    \end{figure}
\end{block}
\end{column}
\begin{column}{5cm}
\begin{block}{Emissões Fugitivas}
{\it "Até $40\%$ de todo o CH4 produzido nos reatores UASB pode
permanecer dissolvido no efluente tratado (SOUZA et al., 2010).
Essa perda representa uma redução no potencial de
recuperação energética. Desse modo, para uma estimativa
mais acurada, essa parcela precisa ser considerada no balaço
de massa de DQO."}
\end{block}
\end{column}
\end{columns}
\end{frame}


\section{Funções e algebra generalizada}

\begin{frame}
\begin{figure}
\includegraphics[width = 7cm]{tsallis1988_resumo}
\caption{Tsallis, C. (1988). Possible generalization of Boltzmann-Gibbs statistics. Journal of Statistical Physics, 52(1–2), 479–487.}
\end{figure}
\begin{columns}
\begin{column}{5cm}
\begin{block}{Tsallis}
\begin{equation}
S_q = \frac{k_B}{q-1}\sum^W_{i=1}p_i(1-p^{q-1}_i) \nonumber
\end{equation}  
\end{block}
\end{column}
\begin{column}{5cm}
\begin{block}{Boltzmann-Gibbs}
\begin{equation}
S_1 = k_B\sum^W_{i=1}p_i\log p_i \nonumber
\end{equation}
\end{block}
\end{column}
\end{columns}
\end{frame}


\begin{frame}{$q$-exponencial}
%\begin{columns}
%		\begin{column}{3.5cm}
A função $q$-exponencial,
 $\exp_q x = [1+(1-q)x]_+^{\frac{1}{1-q}}$,
%
onde o símbolo $[A]_+$ como $[A]_+=A$,  se $A>0$, e $[A]_+\equiv0$ se $A\le0$.
A função inversa da $q$-exponencial é definida por 
%
%\begin{equation}
 $\ln_q x \equiv \frac{x^{1-q}-1}{1-q}$.
%\end{equation}
%	\end{column}
%	\begin{column}{5cm}
           \begin{figure}
            \includegraphics[width = 6.5cm]{_q_exp_all_cases_v01.eps}
	   \end{figure}
%	\end{column}
%\end{columns}
\end{frame}

\begin{frame}{$q$-álgebra}
\begin{figure}
\includegraphics[width = 7cm]{borgesalgebra}
\caption{Borges, E. (2004). A possible deformed algebra and calculus inspired in nonextensive thermostatistics. Physica A: Statistical Mechanics and Its Applications, 340(1–3), 95–101.}
\end{figure}
\end{frame}

\begin{frame}{$q$-álgebra}
A funções $q$-exponencial e $q$-logarítimo leva
a álgebra $q$-deformada não distributiva.    
Operações algebricas generalizadas 
($q$-adição $x \oplus_q y$, $q$-difença $x \ominus_q y$,
$q$-produto $x \otimes_q y$, $q$-razão $x \oslash_q y$) são definidas como
%
\begin{equation}
\label{eq:qsum}
 x\oplus_q y \equiv x+y+(1-q)xy,
\end{equation}
%
\begin{equation}
\label{eq:qdiff}
 x\ominus_q y \equiv \frac{x-y}{1+(1-q)y}, \quad (y\ne \frac{1}{q-1}),
\end{equation}
%
\begin{equation}
\label{eq:qproduct}
 x\otimes_q y \equiv \left[x^{1-q}+y^{1-q}-1\right]^{\frac{1}{1-q}}_+
 \quad (x,y>0),
\end{equation}
%
\begin{equation}
\label{eq:qratio}
 x\oslash_q y \equiv \left[x^{1-q}-y^{1-q}+1\right]^{\frac{1}{1-q}}_+
 \quad (x,y>0).
\end{equation}
%
Com estas $q$-operações, a $q$-exponencial  segue as propriedades:
%
\begin{eqnarray}
  \begin{array}{lll}
  \exp_q x \exp_q y          &=&\exp_q(x\oplus_qy), 
%  &
  \\
  \exp_q x / \exp_q y         &=&\exp_q(x\ominus_qy),
  \\
  \exp_q x \otimes_q \exp_q y &=&\exp_q(x+y),
%  &
  \\
  \exp_q x \oslash_q \exp_q y &=&\exp_q(x-y).
  \end{array}
\end{eqnarray}
%

\end{frame}

\begin{frame}{$q$-trigonometria}
\begin{figure}
\includegraphics[width = 7cm]{borgestrigon}
\caption{Borges, E. (1998). On a q-generalization of circular and hyperbolic functions. Journal of Physics A: Mathematical and General, 31, 5281.}
\end{figure}
\end{frame}


\begin{frame}{$q$-trigonometria}




\end{frame}



\section{Modelos de Digestão Anaeróbica}
\begin{frame}{Modelos}
\begin{columns}
\begin{column}{5cm}
\begin{block}{Haldane}
\begin{itemize}
\item Acidogênese
\item Metanogênese
\end{itemize}
Sistema de 4 Equações Diferenciais
\end{block}
\end{column}
\begin{column}{5cm}
\begin{block}{ADM1}
\begin{itemize}
\item Hidrólise, 
\item Acidogênese, 
\item Acetogênese e 
\item Metanogênese
\end{itemize}
Sistema de 28 Equações Diferenciais
\end{block}
\end{column}
\end{columns}
\end{frame}

\section{Mapeamento da função objetivo}
\begin{frame}{Mapeamento da função objetivo}
    \begin{figure}[h!]
    \centering
    \begin{minipage}{.5\columnwidth}
       \centering
       \includegraphics[width=0.9\columnwidth]{super_static.eps}
\captionof{figure}{Função objetico em função da taxa de diluição}
\label{fig:super_static}
    \end{minipage}%
    \begin{minipage}{.5\columnwidth}
       \centering
       \includegraphics[width=0.95\columnwidth]{time_response.eps}
       \captionof{figure}{Tempo de respostaADM1}
       \label{response}
    \end{minipage}
    \end{figure}
\end{frame}


\section{Sistemas com Feedback}
\begin{frame}
\begin{block}{Diagrama ES para mapa estático de uma entrada}
\begin{figure}[ht]
\begin{center}
\includegraphics[width=0.65\textwidth]{escdiagram1.jpg}
%\label{escdiagram1}
\end{center}
\end{figure}
 \end{block}
\end{frame}


\section{Proposta de Modificação}

\begin{frame}
\begin{block}{Diagrama ES para mapa estático de uma entrada}
\begin{figure}[ht]
\begin{center}
\includegraphics[width=0.65\textwidth]{diagram_ESCq.png}
%\label{escdiagram1}
\end{center}
\end{figure}
 \end{block}
\end{frame}

\begin{frame}{$sinq(\omega t)$}
\begin{block}{}
\begin{figure}[ht] 
    \includegraphics[width=\linewidth]{sinq.eps} 
    \end{figure}
\end{block}
\begin{columns}
\begin{column}{5cm}
\begin{block}{}
\begin{equation}
\sin_q(x) = \rho_q(x)\sin_1[\phi_q(x)] \nonumber
\end{equation}
\end{block}
\end{column}
\begin{column}{5cm}
\begin{block}{}
\begin{equation}
\rho_q(x)={\exp_q((1-q)x^2)}^{\frac{1}{2}} \nonumber
\end{equation}
\begin{equation}
\phi_q(x) = \frac{\arctan_1[(1-q)x]}{1-q}\nonumber
\end{equation}
\end{block}	
\end{column}
\end{columns}
\end{frame}

\section{Resultados}
\begin{frame}{Concentração de Metano}
\begin{figure}[h!]
\centering
%large_range_methane.jpg
\includegraphics[scale=0.45]{methane_sinq_sin1.eps}
\caption{\textcolor{black}{A curva \textcolor{red}{vermelha} representa o extremum seeking
	com a sintonia de amplitude $a_1 = 2.0 \times 10^{-4} $ ($a_q = 0$), a curva  \textcolor{blue}{azul}  $q = 1.01$; $a_q = 2.0 \times 10^{-4}$ and $a_1 = 0$ e a curva \textcolor{green}{verde} $a_1 = 1.0 \times 10^{-5}$ ; $q = 1.01$ and $a_q = 2.0 \times 10^{-4}$ . Em todos os casos, $\omega = 8 \times 10^{-2} ; \omega_h = 5 \times 10^{-3}$ and $k = 1$.}}
\label{fig:LRmethane}
\end{figure}
\end{frame}

\begin{frame}{Taxa de Diluição}
\begin{figure}[h!]
\centering
%large_range_dilution.jpg
\includegraphics[scale=0.45]{diluicao_sinq_sin1.eps}
\caption{\textcolor{black}{A curva \textcolor{red}{vermelha} representa o extremum seeking
	com a sintonia de amplitude $a_1 = 2.0 \times 10^{-4} $ ($a_q = 0$), a curva  \textcolor{blue}{azul}  $q = 1.01$; $a_q = 2.0 \times 10^{-4}$ and $a_1 = 0$ e a curva \textcolor{green}{verde} $a_1 = 1.0 \times 10^{-5}$ ; $q = 1.01$ and $a_q = 2.0 \times 10^{-4}$ . Em todos os casos, $\omega = 8 \times 10^{-2} ; \omega_h = 5 \times 10^{-3}$ and $k = 1$.}}
\label{fig:LRdilution}
\end{figure}
\end{frame}


\begin{frame}{Pertubações}

\begin{figure}[h!]
    \centering
    \subfloat[$a_1 = 1.0\times 10^{-5}$]{\includegraphics[width=4.5cm]{disturbio_1.eps}}%
    \qquad
    \subfloat[$a_1 = 2.5\times 10^{-5}$]{\includegraphics[width=4.5cm]{disturbio_3.eps}}%
    \qquad
    \subfloat[$a_1 = 5.0\times 10^{-5}$]{\includegraphics[width=4.5cm]{disturbio_2.eps}}%
    \qquad
    \subfloat[Respostas após mudanças do ponto máximo.]{\includegraphics[width=4.5cm]{methane_disturbance.eps}}%
    \qquad
    \label{fig:disturbance}%
\end{figure}

\end{frame}

\begin{frame}{Solução Analítica}

\begin{equation} 
|\theta(t)-\theta^{\ast}| \leq |\theta(0)-\theta^{\ast}| \exp({kf^{"}a^2}t/{2}) \nonumber 
\end{equation}

\begin{equation}
                         		|{\tilde{\theta}_{avg}(\tau)-\theta^{\ast}}| \leq 
	|{\tilde{\theta}_{avg}(0)-\theta^{\ast}}| \exp\left(\tau  kcf^{"}\left(a_q^2 {}_2F_1\left(\frac{1}{q-1},\frac{1}{2};\frac{3}{2}; 
					-(1-q)^2{\tau}^2 \right) + a_1^2 \right)  \right) \nonumber
\end{equation}

	\begin{equation}
			\quad 0 \leq \tau < \tau^{\ast} \,.
     \end{equation}
\end{frame}

\section{Perspectivas}
\begin{frame}{Perspectivas}
	\begin{itemize}
		\item Exploração de questões de atraso da metanogênese em função de inibição ou condições desfavoráveis das etapas de hidrólise e acidogênse;
		\item Avaliação sistemática dos limites de estabilidade analítico dos modelos simplificados como Haldane em função do uso de dithers generalizados 
		\item Proposição de dither alternativo baseado nas propriedades da função $\sin_{2-q}(u) \sin_q(u)$ a qual preserva a amplitude e tem ampliação de período.
		\item Avaliação de sistema de equações diferencias parciais com mudança de fase. 
	\end{itemize}
\end{frame}



\section{Principíos do Extremum Seeking}

\subsection{Definições e Notações}
\begin{frame}
\begin{block}{Sistema não linear}
	\begin{equation}
        \frac{dx}{dt} = f(t,x,\epsilon)
	\end{equation}
\end{block}
\begin{block}{}
onde $x \in  \Re^n$ e $f(t,x,\epsilon)$ é periódico em $t$ com período $T$ e domínio $D\subset\Re$, como segue a Equação~(\ref{eq:esc_a}):
\end{block}
\begin{block}{}
\begin{equation}
\label{eq:esc_a}
f(t + T, x,\epsilon ) = f(t, x, \epsilon), \quad \forall (t, x, \epsilon) \in [0, \infty) \times D \times [0, \epsilon_0 ]\,.
\end{equation}
\end{block}
\end{frame}


\begin{frame}{Sistema médio}
	\begin{block}{}
\begin{equation}
\label{eq:esc_b}
\frac{dx_{avg}}{dt} = f_{avg}(x)\,,
\end{equation}
	\end{block}
	\begin{block}{}
\begin{equation}
\label{eq:esc_c}
 f_{avg}(x) = \frac{1}{T} \int_{0}^{T} f(t,x,\epsilon)d\tau\,.
\end{equation}
	\end{block}
	\begin{block}{}
\begin{equation}
\label{eq:esc_d}
f(t,\epsilon)   \leq  C\epsilon, \quad \forall \epsilon \in [0, \epsilon^{\ast} ], \quad \forall \epsilon \in [t_1 , t_2 ].
\end{equation}
	\end{block}
\end{frame}

\section{ES aplicado ao mapa estático}

\begin{frame}
	\begin{block}{Diagrama ES para mapa estático de uma entrada}
\begin{figure}[ht]
\begin{center}
\includegraphics[width=0.65\textwidth]{escdiagram1.jpg}
%\label{escdiagram1}
\end{center}
\end{figure}
        \end{block}

\end{frame}

\begin{frame}
\begin{columns}
	\begin{column}{5cm}
\begin{block}{}
\begin{equation}
\label{eq:esc_e}
y:= f(\theta) = f^{\ast} + \frac{f^{''}}{2}(\theta - \theta^{\ast})\,.
\end{equation}
\end{block}
\begin{block}{}
\begin{itemize}
\item $\theta^{\ast}$ - Valor ótimo desconhecido;
\item $\hat{\theta}(t)$ - Estimativa em tempo real de $\theta^{\ast}$;
\item $\theta(t)$ - Valor atual da entrada do mapa
\end{itemize}
\end{block}
\end{column}
	\begin{column}{5cm}
	\begin{block}{Diagrama ES para mapa estático de uma entrada}
\begin{figure}[ht]
\begin{center}
\includegraphics[width=0.65\textwidth]{escdiagram1.jpg}
\label{escdiagram1}
\end{center}
\end{figure}
        \end{block}
\end{column}
\end{columns}
\end{frame}

\begin{frame}{Reprodução do artigo}
	\begin{block}{Computers and Chemical Engineering 75 (2015) 49–59}
          
	\end{block}
\end{frame}

\begin{frame}{Perspectivas}

\begin{itemize}
\item Identificação de conjuntos de parametrização típicos do dither generalizado. 
\item Desenvolvimento de estruturas de auto-sintonia
\item MPC econômico + q-ESC 
\item \href{DeepONet}{https://www.quantamagazine.org/latest-neural-nets-solve-worlds-hardest-equations-faster-than-ever-before-20210419} 
\end{itemize}

\end{frame}

%###################################################################################

\section{Funções e algebra generalizada}

\begin{frame}{$q$-exponencial}
	\begin{columns}
		\begin{column}{5cm}
A função $q$-exponencial é definida por:
%
\begin{equation}
\label{eq:qexp}
 \exp_q x = [1+(1-q)x]_+^{\frac{1}{1-q}}\nonumber \,,
\end{equation}
%
onde o símbolo $[A]_+$ como $[A]_+=A$,  se $A>0$, e $[A]_+\equiv0$ se $A\le0$.
A função inversa da $q$-exponencial é definida por 
%
\begin{equation}
\label{eq:qlogc2}
 \ln_q x \equiv \frac{x^{1-q}-1}{1-q} \nonumber\,.
\end{equation}
	\end{column}
	\begin{column}{5cm}
           \begin{figure}
            \includegraphics[width = 4.5cm]{_q_exp_all_cases}
	   \end{figure}
	\end{column}
\end{columns}
\end{frame}

% \begin{frame}{$q$-álgebra}
% A funções $q$-exponencial e $q$-logarítimo leva
% a álgebra $q$-deformada não distributiva.    
% Operações algebricas generalizadas 
% ($q$-adição $x \oplus_q y$, $q$-difença $x \ominus_q y$,
% $q$-produto $x \otimes_q y$, $q$-razão $x \oslash_q y$) são definidas como
% %
% \begin{equation}
% \label{eq:qsum}
%  x\oplus_q y \equiv x+y+(1-q)xy,
% \end{equation}
% %
% \begin{equation}
% \label{eq:qdiff}
%  x\ominus_q y \equiv \frac{x-y}{1+(1-q)y}, \quad (y\ne \frac{1}{q-1}),
% \end{equation}
% %
% \begin{equation}
% \label{eq:qproduct}
%  x\otimes_q y \equiv \left[x^{1-q}+y^{1-q}-1\right]^{\frac{1}{1-q}}_+
%  \quad (x,y>0),
% \end{equation}
% %
% \begin{equation}
% \label{eq:qratio}
%  x\oslash_q y \equiv \left[x^{1-q}-y^{1-q}+1\right]^{\frac{1}{1-q}}_+
%  \quad (x,y>0).
% \end{equation}
% %
% Com estas $q$-operações, a $q$-exponencial  segue as propriedades:
% %
% \begin{eqnarray}
%   \begin{array}{lll}
%   \exp_q x \exp_q y          &=&\exp_q(x\oplus_qy), 
% %  &
%   \\
%   \exp_q x / \exp_q y         &=&\exp_q(x\ominus_qy),
%   \\
%   \exp_q x \otimes_q \exp_q y &=&\exp_q(x+y),
% %  &
%   \\
%   \exp_q x \oslash_q \exp_q y &=&\exp_q(x-y).
%   \end{array}
% \end{eqnarray}
% %
% 
% \end{frame}

\section{Interações de Curto Alcance}
\subsection{Potencial de Lennard Jones}
\begin{frame}
O potencial de Lennard-Jonnes contabiliza as contribuições das interações de moléculas 
apolares dividindo em duas parcelas que correspondem às forças repulsivas e atrativas: 
\begin{equation}
	\Gamma_{total} = \Gamma_{repulsiva} + \Gamma_{atrativa} = \frac{A}{r^n} - \frac{B}{r^m}\,.
\end{equation}
	\begin{itemize}
	\item Forças eletrostáticas entre partículas carregadas e dipolos permamente; 
	\item Forças de indução entre dipolo permanente e dipolo induzido;
	\item Forças de atração (forças de dispersão) e repulsão entre moléculas apolares;
	\item forças específicas para associação e formação de complexos
	\end{itemize}
\end{frame}

%\begin{frame}{Não extensividade}
%
%\end{frame}
%
%
%\begin{frame}{Relações Termodinâmicas}
%
%
%\end{frame}

\subsection{Teoria de soluções}

%\subsection{Guggeinheim}

\subsection{Teoria de Scatchard-Hildebrand}
\begin{frame}{Teoria de Scatchard-Hildebrand}
	\begin{itemize}
		\item Solução Regular - desaparecimento da entropia de mistura para $T$ e $V$ constantes.
		\item Densidade de energia coesiva $c \equiv \Delta u^v / v^L$
		\item Definição de regra de mistura binária para
	\end{itemize}
	\begin{columns}
		\begin{column}{6cm}
	\begin{block}{Mistura Binária}
	\begin{equation}
		-(u^{L} - u^{ig} )  = \frac{\sum_i^2 \sum_j^2 c_{ij}v_iv_jx_ix_j}{\left(\sum_k^2 x_k v_k \right)}
	\end{equation}
	\end{block}
		\end{column}
		\begin{column}{4cm}
			\begin{block}{Fração do volume}
                 \begin{equation}
                     \Phi_i = \frac{x_iv_i}{\sum_k^2 x_k v_k}
		 \end{equation}
			\end{block}
		\end{column}
	\end{columns}
	\begin{block}{ }
	$-(u^{L} - u^{ig} )  =  (x_1v_1 + x_2v_2)[c_{11}\Phi_{1}^2 + 2c_{12}\Phi_{1}\Phi_{2} + c_{22}\Phi_{2}^{2}]$
	\end{block}
\end{frame}

\subsection{Modelo de Wilson}

\begin{frame}{Modelo de Wilson}
%
\begin{equation}
\label{eq:ge-qwilson}
 \frac{g^E}{RT} = \sum_i^c x_i \ln \frac{\xi_{ii}}{x_i},
\end{equation}
%
onde $\xi_{ii}$  é a fração volumétrica local do compoenente $i$
na vizinhança de outra molécula $i$ 
( O modelo de Flory-Huggins para soluções poliméricas atérmicas 
usa a fração de segmento global $\xi_i$ no lugar de 
$\xi_{ii}$, na Eq.~(\ref{eq:ge-qwilson})).
%
A fração volumétrica local é dada por 
%
\begin{equation}
\label{eq:xi_ii}
 \xi_{ii}=\frac{x_{ii} v_i}{\sum_j^c x_{ji} v_j},
\end{equation}
%
\end{frame}

\begin{frame}{Composição Local}
	\begin{block}{Composição local}
		\begin{equation}
		\frac{x_{ij}}{x_{jj}}=\frac{x_i}{x_j} \frac{\exp(-\frac{a_{ij}}{RT})}{\exp(-\frac{a_{jj}}{RT})}\,,
		\end{equation}
	\end{block}
onde  $x_i$ 
é a fração molar {\it bulk} da espécie $i$, e 
$x_{ij}$ é a fração molar local da espécie $i$
nas vizinhanças da molécula $j$.
%\begin{itemize}
%	\item Comentários
%\end{itemize}
\end{frame}


\begin{frame}{Dependência de $A_{ij}$ com temepratura}
%
\begin{equation}
\label{eq:A1}
 A_{ij}=\exp\left( - \frac{\Delta a_{ij}}{RT}\right),
\end{equation}
com $\Delta a_{ij}=a_{ij}-a_{jj}$, e
$a_{ij}$ 
é a energia potencial molar de interações entre espécies 
$i$ e $j$, com $a_{ji}=a_{ij}$. 
%
%
% empirical changes for the temperature dependence
%$\Delta a_{ij}$ é originalmente assumido constante. 
%Extensões do modelos relaxam esta hipótese, e 
%consideram 
	\begin{columns}
		\begin{column}{5cm}

			\begin{block}{Renon (1968)}
$\Delta a_{ij}=\Delta a_{ij}(T)$,
%de acordo com várias funções, por exemplo, uma relação linear  \cite{Renon1968}, 
			\end{block}
			\begin{block}{Thomsen}
%\begin{equation}
%\label{eq:thomsen}
$ \Delta a_{ij}=\Delta a_{ij,0}+\Delta a_{ij,1}(T-T_0)$
%\end{equation}
			\end{block}
%
			\begin{block}{Anderson-Prausnitz}			
%o inverso da temperatura absoluta~\cite{anderson-prausnitz},
%\begin{equation}
%\label{eq:anderson}
$ \Delta a_{ij}=\Delta a_{ij,0}+\frac{\Delta a_{ij,1}}{T}$
%\end{equation}
			\end{block}
%
%uma combinação de ambas as relações linear e inversa~\cite{escobedo-sandler},
		\end{column}
		\begin{column}{5cm}
			\begin{block}{Escobedo-Sandler}
%			\begin{equation}
$ \Delta a_{ij}=\Delta a_{ij,0}+\Delta a_{ij,1} T
 +\frac{\Delta a_{ij,2}}{T}$
%\end{equation}
			\end{block}
			%
%ou, então \cite{demirel-gecegormez-paksoy-p1:1992},
			\begin{block}{Dimerel et al.}
%\begin{equation}
%\label{eq:schmidt}
$ \Delta a_{ij}=\Delta a_{ij,0}+\frac{\Delta a_{ij,1}}{T-T_0}$
%\end{equation}
			\end{block}
%
%ou mesmo com o termo logarítimo~\cite{larsen:1987}:
			\begin{block}{Larsen}			
%\begin{eqnarray}
%\label{eq:larsen:1987}
$ \Delta a_{ij}=\Delta a_{ij,0}+\Delta a_{ij,1}(T-T_0) $
% \nonumber \\
$ +\Delta a_{ij,2} \left(T\ln \frac{T_0}{T} + T - T_0 \right)$
%\end{eqnarray}
			\end{block}
%
		\end{column}
	\end{columns}
\end{frame}

\begin{frame}{Composição local generalizada}
\begin{equation}
\frac{x_{ij}}{x_{jj}}=
\label{eq:xijxjj}
                       \frac{x_i}{x_j}
                       \exp_{q_{ij}}\left(-\frac{a_{ij}}{RT}\right) 
                       \oslash_{q_{ij}}
                       \exp_{q_{ij}}\left(-\frac{a_{jj}}{RT}\right).
\end{equation}
%
Substituição da Eq.~(\ref{eq:xijxjj}) 
na condição de normalização  
$\sum_i^c x_{ij}=1$ ($c$ é o número de espécies químicas) resulta
%
\begin{equation}
\label{eq:xij}
 x_{ij}=\frac{x_i A_{q,ij}}{\sum_k^c x_k A_{q,kj}},
\end{equation}
%
com o parâmetro $A_{q,ij}$ dado por%
\footnote{Nós adotamos o símbolo $A_{q,ij}$, ao invés de 
$A_{q_{ij},ij}$, para evitar uma notação desnecessária.}
%
\begin{equation}
\label{eq:Aq}
 A_{q,ij} \equiv \exp_{q_{ij}}\left(-\frac{\Delta a_{ij}}{RT}\right),
\end{equation}
%
com $\Delta a_{ij}=a_{ij}-a_{jj}$. 
A simetria das interações implica  $a_{ji}=a_{ij}$,
e nós assumimos, por simplicidade, $q_{ji}=q_{ij}$.
O caso limite $q_{ij}\to 1$ 
recupera o parâmetro usual
$A_{ij}\equiv A_{1,ij}$,
%
\begin{equation}
\label{eq:A1}
 A_{1,ij}=\exp_1\left( - \frac{\Delta a_{ij}}{RT}\right),
\end{equation}
%
%	Eq.~(\ref{eq:A1}).
\end{frame}


\begin{frame}
%
\begin{figure}[htb]
\begin{center}
 \includegraphics[width=0.75\columnwidth,keepaspectratio,clip]{A12vsa12.eps}
\end{center}
\caption{\label{fig:A12-a12}%
         O parâmetro  $A_{q,12}$, Eq.~(\protect\ref{eq:Aq}), como uma função do 
          parâmetro de interação binário $\Delta a_{12}$ (com $T=298.15$~K).
         Com $q_{12}=0.5$ (pontilhada e tracejada), $q_{12}=1$ (sólida), 
         e $q_{12}=1.5$ (tracejada).
         ponto-ponto tracejada as curvas usam Eq.  de Anderson para 
         $\Delta a_{12,1}=5\times10^5$~J~K/mol.
}
\end{figure}
%
\end{frame}

\begin{frame}

\begin{figure}
\begin{center}
 \includegraphics[width=0.45\columnwidth]{A12a5000J.eps}
% 
 \includegraphics[width=0.45\columnwidth]{A12a-5000J.eps}
\end{center}
\caption{\label{fig:A12-1suT}%
         O parâmetro $A_{q,12}$, Eq.~(\protect\ref{eq:Aq}),
         como uma função do inverso da temperatura (convencionalmente escalado),
         para dois valores típicos de  $\Delta a_{12}$.
         O caso do modelo ordinário $q_{12}=1$  aparece como linha cheia 
	 nestes gráficos semi-log. 
         $q_{12}>1$ ($q_{12}<1$) apresenta concavidade positiva (negativa).
}
\end{figure}
\end{frame}

\begin{frame}
\begin{figure}
\begin{center}
 \includegraphics[width=0.5\columnwidth]{A12a5000J.eps}
\end{center}
\caption{\label{fig:A12-1suT}%
         O parâmetro $A_{q,12}$, Eq.~(\protect\ref{eq:Aq}),
         como uma função do inverso da temperatura (convencionalmente escalado),
         para dois valores típicos de  $\Delta a_{12}$.
         O caso do modelo ordinário $q_{12}=1$  aparece como linha cheia 
	 nestes gráficos semi-log. 
         $q_{12}>1$ ($q_{12}<1$) apresenta concavidade positiva (negativa).
}
\end{figure}
\end{frame}

\begin{frame}{Coeficiente de Atividade a diluição infinita}
%
\begin{equation}
 \frac{g^E}{RT}=-\sum_i^c x_i \ln\left(\sum_j^c x_j \Lambda_{q,ij}\right),
\end{equation}
%
com $\Lambda_{q,ij}=(v_j/v_i)A_{q,ij}$.
%
O coeficiente de atividade é formalmente dado pela mesma expressão  do modelo
original ($q_{ij}=1$) , mas com o parâmetro $\Lambda_{q,ij}$ dependente de $q$ :
%
\begin{equation}
 \ln \gamma_i = -\ln\left(\sum_j^c x_j \Lambda_{q,ij}\right) + 1 
         - \sum_j^c \frac{x_j \Lambda_{q,ji}}{\sum_k^c x_k \Lambda_{q,jk}}.
\end{equation}
%
\end{frame}

\begin{frame}{Resultado - Modelo de Wilson}
	\begin{block}{IDAC q-Wilson mistura binária}
           $\ln \gamma_1^\infty = \ln \Lambda_{q,12} - \Lambda_{q,21} + 1.$
	\end{block}
\begin{figure}
\begin{center}
 \includegraphics[width=0.75\columnwidth,keepaspectratio,clip]{ginft1a12+10000Ja21-5600J.eps}
\end{center}
\caption{\label{fig:ginfty}%
         $\Delta a_{12}=10^4$~J/mol e $\Delta a_{21}=-5600$~J/mol
         ($v_1=v_2$). 
}
\end{figure}
\end{frame}

\begin{frame}{IDAC - etanol(1)-tolueno(2)}
\begin{figure}
\begin{center}
 \includegraphics[width=0.75\columnwidth]{ginf-T-toluene-ethanol.eps}
\end{center}
\caption{\label{fig:ethanol}%
         (a) Linhas tracejadas: 
             $\Delta a_{12}= 6880.0$~J/mol, $\Delta a_{21}=-582.0$~J/mol;
             linhas sólidas: 
         $q_{12}=0.83$, 
         $\Delta a_{12}=10267.0$~J/mol, $\Delta a_{21}=-5179.7$~J/mol.
         }
\end{figure}
\end{frame}

		\begin{frame}{IDAC - etanol(1)-decano(2)}
\begin{figure}
\begin{center}
 \includegraphics[width=0.75\columnwidth]{ginf-T-decane-ethanol.eps}
\end{center}
\caption{\label{fig:ethanol2}%
         (b) Linhas tracejadas: 
             $\Delta a_{12}= 8773.6$~J/mol, $\Delta a_{21}=-26.6$~J/mol;
             linha sólida: 
         $q_{12}=0.7$, $\Delta a_{12}=6500.0$~J/mol, 
         $\Delta a_{21}=-2353.7$~J/mol.}
\end{figure}
\end{frame}

\begin{frame}{Equilíbrio líquido-vapor}
A lei de Raoult modificada é:	%
\begin{equation}
	\label{eq:mraoultlaw}
	x_i \gamma_i(T,P,x) P_{i}^{vap}(T)  = y_iP\,,
\end{equation}
que sujeita ao vínculo de normalização da fração molar da fase gasosa $\sum_i^{c} y_i $, 
permite o isolamento da pressão $P = \sum_i^{c} x_i \gamma_i(T,P,x) P_{i}^{vap}(T)$. 
	\begin{block}{função objetivo}
		$f_{obj} = \sum_k^{N_{p}}\sum_i ((\gamma_{ik} - \gamma_{ik}^{exp})/ \gamma_{ik}^{exp})^2$,
	\end{block}
\begin{figure}
\includegraphics[width=0.4\columnwidth]{gamma_vs_x_acetone_methanol.eps}
\includegraphics[width=0.4\columnwidth]{P_vs_x_acetone_methanol.eps}
        \end{figure}
\end{frame}

\subsection{Modelo NRTL}
\begin{frame}{NRTL}
\begin{eqnarray}
\label{eq:ge-nrtl} 
 g^E &=& \sum_i^c x_i \sum_j^c x_{ji} \Delta a_{ji} \nonumber \\
     &=& \sum_i^c x_i \frac{\sum_j^c x_j A_{q,ji} \Delta a_{ji}}
                           {\sum_k^c x_k A_{q,ki}}.
\end{eqnarray}
%
O modelo NRTL introduz um parâmetro de randomicidade 
 $\alpha_{ij}=\alpha_{ji}$,
apenas o parâmetro $A_{q,ij}$ é dado pela variação da Eq.~(\ref{eq:Aq}), 
nomeadamente 
$ A_{q,ij} = \exp_{q_{ij}}(-\alpha_{ij}\frac{\Delta a_{ij}}{RT})$.
%
A expressão para o coeficiente de atividade para o componente $i$ é 
formalmente o mesmo que o original do modelo NRTL ($q_{ij}=1$), 
apenas substituindo o parâmetro usual $A_{ij}$ por $A_{q,ij}$.
Este procedimento não é válido para encontrar entropia molar de excesso e
entalpia molar de excesso, devido a:  %Eq.~(\ref{eq:dAqdT}).
%
\begin{equation}
\label{eq:dAqdT}
 \frac{dA_{q,ij}}{dT}= A_{q,ij}^{q_{ij}} \frac{\Delta a_{ij}}{RT^2}.
\end{equation}
\end{frame}



\section{Interações de Longo Alcance}

\subsection{Equação de Poisson}

\subsubsection{Soluções Analíticas}
\begin{frame}{Poisson-Boltzmann}
	Equação de Poisson:
	\begin{equation}
           \nabla^2 \psi \approx -\frac{\rho}{\epsilon \epsilon_0}
        \end{equation}
considerando a densidade de carga segundo a distribuição de Boltzmann;
\begin{equation}
	\nabla^2 \psi(r) = -\frac{1}{\epsilon_{r}\epsilon_{0}}{\sum_{i}^{n}z_i |e|N_{i}\exp\left(-\frac{z_i |e|\psi(r)}{{k_b}T}\right)}
\end{equation}

\begin{equation}
	\nabla^2 \psi(r) = \frac{e^{2} \sum_{i}^{n} N_i z_{i}^{2}}{ \epsilon \epsilon_0 k_b T }{\psi(r)} 
\end{equation}

	\begin{equation}
		\kappa^{2} =   \frac{e^{2} \sum_{i}^{n} N_i z_{i}^{2}}{ \epsilon \epsilon_0 k_b T }
	\end{equation}


	Parte da solução da equação diferencial linear  acima é dada por:

	\begin{equation}
\psi(r) = \frac{A}{r}\exp(-\kappa r)
\end{equation}

\end{frame}

\subsection{Coeficiente de Atividade Iônico Médio}
\begin{frame}{Coeficiente de Atividade Iônico Médio}
Ao considerarmos apenas os primeiros termos da expansão em série da equação 
\begin{equation}
\psi(r) = \frac{A}{r}\exp(-\kappa r),
\end{equation}
	obtemos:
	\begin{equation}
            \psi(r) = \frac{A}{r} - {A\kappa}.
	\end{equation}
	Para o limite de $lim_{N_i\to 0} {\kappa} = 0 $, e o potencial na superficie para um único 
	íon é	dado por $\psi_s(a) = \frac{z_{+}e }{4\pi\epsilon_r \epsilon_0 a} $, logo,
	\begin{equation}
		\label{pdhll}
		\psi(r) =  \frac{z_{+}e }{4\pi\epsilon_r \epsilon_0 r} -   \frac{z_{+}e \kappa}{4\pi\epsilon_r \epsilon_0 } 
	\end{equation}
\end{frame}

\begin{frame}{G\"{u}ntelberg-M\"{u}ller}
	  A aplicação do processo de carga de G\"{u}ntelberg-M\"{u}ller à  equação~\ref{pdhll} a menos do íon central $j$
	  nos fornece a energia extra e um íon devido à atmosfera de interação resultando em 
	  \begin{equation}
		  W_j = \int^{q}_{0} (\psi(a)- \psi_s(a))dq
	  \end{equation}
            sendo a energia de Helmholtz  $F^{el} = W_j$, logo: 
	\begin{equation}
		\log_{10} \gamma^{\pm} = -\log_{10}(\exp(1)) \frac{e^2 \kappa }{8 \pi k_b T}z_{+}z_{-}
	\end{equation}
%\begin{equation}
%\nabla^2 \psi(r) = -\frac{4\pi}{\epsilon_{r}\epsilon_{0}}\rho(r)
%\end{equation}

%\begin{equation}
%\nabla^2 \psi(r) = \kappa^2\psi(r)
%\end{equation}
%
%\begin{equation}
%\kappa^2 = \frac{\sum_{i}^{n}N_i{z_i}^2 {|e|}^2}{V\epsilon_{r}\epsilon_{0}{k_b}T}
%\end{equation}
\end{frame}

\begin{frame}{Debye-Huckel}
A indução elétrica da carga sobre o íon $\alpha$ na direção radial é dada por:
	\begin{equation}
		-\epsilon_r \epsilon_0 \frac{\partial \psi}{\partial r } =
		 \frac{\epsilon_r \epsilon_0 A}{r^2} \exp(-\kappa r)(1+\kappa r ) \quad (r\geq a )    
	\end{equation}

	\begin{equation}
		\frac{\epsilon_r \epsilon_0 A}{a^2} \exp(-\kappa a)(1+\kappa a )    =   \frac{z_{\alpha}|e|}{a^2}
	\end{equation}
	logo,
	\begin{equation}
		\psi(r) = \frac{z_{\alpha}|e|}{4\pi \epsilon_r \epsilon_0 r }\frac{\exp(-\kappa (r-a))}{1+\kappa a } 
        \end{equation}

	\begin{equation}
		\psi_s(a) = \frac{z_{\alpha}|e|}{4\pi \epsilon_r \epsilon_0}\frac{1}{1+\kappa a } 
        \end{equation}

        \begin{equation}
		\psi_{\alpha}(a) = -\frac{z_{\alpha}|e|}{4\pi \epsilon_r \epsilon_0}\frac{\kappa}{1+\kappa a } 
        \end{equation}
\end{frame}

\begin{frame}{G\"{u}ntelberg-M\"{u}ller}
	\begin{eqnarray}
		F^{el} = \int_{0}^{1} \sum_{\alpha}\psi_{\alpha}(\lambda)z_{\alpha}d\lambda = 
		\int_{0}^{1} \sum_{i} N_i\psi_{i}(\lambda)z_{i}|e|d\lambda \\\nonumber
		= -\sum_{i} \frac{N_{i}z_{i}^{2}|e|^2\kappa}{4\pi \epsilon_r \epsilon_0 }\tau(\kappa a) 
	\end{eqnarray}
	onde, $\tau(\kappa a) = \int_{0}^{1}\frac{\lambda^2}{1+\lambda \kappa a} d \lambda $.
Pela definição de energia parcial molar: 
	\begin{equation}
		\mu^{el}_{i} = \frac{\partial G }{\partial N_i} =
		-\frac{z_i^2 |e|^2 }{2 4\pi \epsilon_r \epsilon_0 }\frac{\kappa }{1+\kappa a},
	\end{equation}
ainda é possível considerar a parcela correspondente da interação iônica com solvente:
	\begin{equation}
		\mu^{el}_{i} = \frac{\partial G }{\partial N_i} =
		-\frac{z_i^2 |e|^2 }{8\pi \epsilon_r \epsilon_0 }\frac{\kappa }{1+\kappa a}
		+\frac{\kappa^3V_i}{24 \pi N_A }\sigma(\kappa a) .
	\end{equation}
\end{frame}


\begin{frame}{$q$-Debye-Huckel}
	A generalização da solução de Debye-H\"{u}ckel:
\begin{equation}
	\label{eq:sph}
	\psi(r) = \frac{A}{r}\exp_q(-\kappa r), 
\end{equation}
	\begin{equation}
		\psi_q(r) = \frac{z_{\alpha}|e|}{4\pi \epsilon_r \epsilon_0 (1+\kappa a( \exp_q(-\kappa a))^{q-1})r}\frac{\exp_q(-\kappa r)}{\exp_q(-\kappa a)} 
        \end{equation}
	\begin{figure}
         \includegraphics[width=6cm]{_q_exp_all_cases_v01.eps}
	\end{figure}

\end{frame}

\begin{frame}{Exemplos da solução de $q$-Debye-H\"{u}ckel}
\begin{figure}
\centering
  \includegraphics[width=7.5cm]{esferico_q}
	\caption{Influência do parâmetro $q$ na Equação $q$-Debye-H\"{u}ckel em escala {\it linear}.}
\end{figure}
\end{frame}

\begin{frame}{Exemplos da solução de $q$-Debye-H\"{u}ckel}
\begin{figure}
\centering
  \includegraphics[width=7.5cm]{esferico_q_log_linear}
	\caption{Influência do parâmetro $q$ na Equação $q$-Debye-H\"{u}ckel em escala {\it log-linear}.}
\end{figure}
\end{frame}

\begin{frame}{Exemplos da solução de $q$-Debye-H\"{u}ckel}
\begin{figure}
\centering
  \includegraphics[width=7.5cm]{esferico_q_log_log}
	\caption{Influência do parâmetro $q$ na Equação $q$-Debye-H\"{u}ckel em escala {\it log-log}.}
\end{figure}
\end{frame}

\begin{frame}{Coeficiente de atividade -  $q$-Debye-H\"{u}ckel}
	A generalização da solução de Debye-H\"{u}ckel:
	\begin{equation}
		\psi_q(r) = \frac{z_{\alpha}|e|}{4\pi \epsilon_r \epsilon_0 (1+\kappa a( \exp_q(-\kappa a))^{q-1})r}\frac{\exp_q(-\kappa r)}{\exp_q(-\kappa a)} 
        \end{equation}
	Definindo $\psi_{q,\alpha}$
	\begin{equation}
		\psi_{q,\alpha} = \frac{z_{\alpha}|e|}{4\pi \epsilon_r \epsilon_0 (1+\kappa a( \exp_q(-\kappa a))^{q-1})a} 
		- \frac{z_{\alpha}|e|}{ 4\pi \epsilon_r \epsilon_0 a}.  
	\end{equation}
	A energia livre é obtida pelo processo de carga de G\"{u}ntelberg.
	\begin{eqnarray}
		F^{el} = \int_{0}^{1} \sum_{\alpha}\psi_{q,\alpha}(\lambda)z_{\alpha}d\lambda 
	\end{eqnarray}
	A integração da equação acima foi realizada numericamente bem como o procedimento para calcular o potencial 
	$\mu^{el} = \frac{\partial G }{\partial N_i}$.  
\end{frame}

\begin{frame}
\begin{figure}
\centering
  \includegraphics[width=7.5cm]{KCl_activity}
	\caption{Influência do parâmetro $q$ na Equação $q$-Debye-H\"{u}ckel em escala {\it log-log}.}
\end{figure}

\end{frame}

\begin{frame}
\begin{figure}
\centering
  \includegraphics[width=7.5cm]{KCl_hamer_activity_v00.eps}
	\caption{Influência do parâmetro $q$ na Equação $q$-Debye-H\"{u}ckel em escala {\it log-log}.}
\end{figure}
\end{frame}

\begin{frame}
\begin{figure}
\centering
  \includegraphics[width=10.0cm]{activity_coef_all.eps}
	\caption{Influência do parâmetro $q$ na Equação $q$-Debye-H\"{u}ckel em escala {\it log-linear}.}
\end{figure}
\end{frame}








\subsection{Tensão Superficial}

\begin{frame}
	\begin{figure}
 \centering
  \includegraphics[width=10.0cm]{abstract_Onsager.jpg}
		\caption{Resumo do artigo do Onsager e Samaras (1934).}
	\end{figure}
\end{frame}

\begin{frame}
	\begin{columns}
		\begin{column}{3.5cm}
	\begin{block}{Onsager Samaras}
	$	u^{im} = \frac{e^2}{16\pi \epsilon z } exp(-2z/L_D )$
	\end{block}
			\begin{block}{Tensão superficial - OS}
			       $\Delta \sigma = \alpha \times c \ln(\frac{\beta}{c})$ 
			\end{block}
		\end{column}
		\begin{column}{6.0cm}
	\begin{figure}
 \centering
  \includegraphics[width=5.0cm]{diagram_esq_onsager}
		\caption{Diagrama esquemático Onsager e Samaras (1934) .}
	\end{figure}

		\end{column}
	\end{columns}
%        \begin{block}{Proposta} 
%	\begin{equation}
%		u^{im}_q(r) = \frac{(z_{\alpha}|e|)^2}{4\pi \epsilon_r \epsilon_0 (1+\kappa a( \exp_q(-2\kappa a))^{q-1})4r}\frac{\exp_q(-2\kappa r)}{\exp_q(-2\kappa a)} 
%        \end{equation}
%	\end{block}
\end{frame}

\begin{frame}{}
%	\begin{block}{Onsager Samaras}
%	\begin{equation}
%		u_{im} = \frac{e^2}{16\pi \epsilon z } exp(-2z/L_D )
%	\end{equation}
%	\end{block}
         \begin{block}{Schmutzer's function}
	\begin{eqnarray}
		\Gamma_i^{z=0} = c\int_{R_i}^{\infty} (exp(-u^{im}/{k_B T})-1)dz - cR_i \\\equiv  \frac{L_B^{-1/2} c^{1/2}}{2\sqrt{2}}f_s(a_Dc^{1/2},R_i/L_D)
	\end{eqnarray}
	 \end{block}
	 \begin{equation}
               f_s = \int_{2R_i/L_D}^{\infty} (\exp(-a_D c^{1/2}\exp(-Z)/Z) -1)dZ - \frac{2R_i}{L_D}
	 \end{equation}
\end{frame}

	\begin{frame}{Slavchov 2012}
		\begin{columns}
			\begin{column}{5cm}
	\begin{figure}
         \centering
          \includegraphics[width=5cm]{slavchov_model.jpg}
	\caption{Camada da interface composta pela espessura de uma molécula de água.}
	\end{figure}
			\end{column}
			\begin{column}{5cm}
	\begin{eqnarray}
		\Delta \sigma = -2\int_{0}^{C_m} \Gamma_i d\mu_{el} \nonumber \\
		 = -\frac{T}{\sqrt(2)} \int_{0}^{C_m}\frac{f_s(a_Dc^{1/2},R_i/L_D)}{1-v_{el}c} \nonumber \\
		 \times \frac{L_B^{-1/2c^{1/2} }}{C_m}\left( 1+c_m \frac{d\ln\gamma_{\pm}}{dc_m}\right)dc_m \nonumber
	\end{eqnarray}
		\begin{equation}
                 \log_{10}\gamma_{\pm} = -\frac{A_D \sqrt{c_m}}{1+B\sqrt{c_m}} +\sum\beta_jc_m^j \nonumber
		\end{equation}
			\end{column}
		\end{columns}
\end{frame}


\begin{frame}
	\begin{columns}
		\begin{column}{3.5cm}
	\begin{block}{Onsager Samaras}
	$	u^{im} = \frac{e^2}{16\pi \epsilon z } exp(-2z/L_D )$
	\end{block}
		\end{column}
		\begin{column}{6.0cm}
	\begin{figure}
 \centering
  \includegraphics[width=5.0cm]{diagram_esq_onsager}
		\caption{Diagrama esquemático Onsager e Samaras (1934) .}
	\end{figure}

		\end{column}
	\end{columns}
        \begin{block}{Proposta} 
	\begin{equation}
		u^{im}_q(r) = \frac{(z_{\alpha}|e|)^2}{4\pi \epsilon_r \epsilon_0 (1+\kappa a( \exp_q(-2\kappa a))^{q-1})4r}\frac{\exp_q(-2\kappa r)}{\exp_q(-2\kappa a)} 
        \end{equation}
	\end{block}

\end{frame}


%\begin{frame}
%\begin{columns}
%\begin{column}{4cm}
%	\begin{block}{NaCl}
%    \begin{figure}
%     \includegraphics[scale=0.15]{sem_epsl_conc_cmodel_NaCl_Slavchov_2012.eps}
%     \end{figure}
%	\end{block}
%	\begin{block}{NaCl}
%    \begin{figure}
%     \includegraphics[scale=0.5]{apendice_NaCl}
%     \end{figure}
%	\end{block}
%\end{column}
%\begin{column}{4cm}
%	\begin{block}{NaBr}
%    \begin{figure}
%     \includegraphics[scale=0.15]{sem_epsl_conc_cmodel_NaBr_Slavchov_2012.eps}
%     \end{figure}
%	\end{block}
%	\begin{block}{NaBr}
%    \begin{figure}
%     \includegraphics[scale=0.5]{apendice_NaBr}
%     \end{figure}
%	\end{block}
%\end{column}
%\begin{column}{4cm}
%	\begin{block}{NaI}
%    \begin{figure}
%	    \includegraphics[scale=0.15]{sem_epsl_conc_cmodel_NaI_Slavchov_2012.eps}
%     \end{figure}
%	\end{block}
%	\begin{block}{NaI}
%    \begin{figure}
%     \includegraphics[scale=0.5]{apendice_NaI}
%     \end{figure}
%	\end{block}
%\end{column}
%\end{columns}
%\end{frame}
%
%
%\begin{frame}
%\begin{columns}
%\begin{column}{4cm}
%	\begin{block}{KCl}
%    \begin{figure}
%     \includegraphics[scale=0.15]{sem_epsl_conc_cmodel_KCl_Slavchov_2012.eps}
%     \end{figure}
%	\end{block}
%	\begin{block}{KCl}
%    \begin{figure}
%     \includegraphics[scale=0.5]{apendice_KCl}
%     \end{figure}
%	\end{block}
%\end{column}
%\begin{column}{4cm}
%	\begin{block}{KBr}
%    \begin{figure}
%     \includegraphics[scale=0.15]{sem_epsl_conc_cmodel_KBr_Slavchov_2012.eps}
%     \end{figure}
%	\end{block}
%	\begin{block}{KBr}
%    \begin{figure}
%     \includegraphics[scale=0.5]{apendice_KBr}
%     \end{figure}
%	\end{block}
%\end{column}
%\begin{column}{4cm}
%	\begin{block}{KI}
%    \begin{figure}
%	    \includegraphics[scale=0.15]{sem_epsl_conc_cmodel_KI_Slavchov_2012.eps}
%     \end{figure}
%	\end{block}
%	\begin{block}{KI}
%    \begin{figure}
%     \includegraphics[scale=0.5]{apendice_KI}
%     \end{figure}
%	\end{block}
%\end{column}
%\end{columns}
%\end{frame}



\begin{frame}
\begin{columns}
\begin{column}{4cm}
	\begin{block}{NaCl}
    \begin{figure}
     \includegraphics[scale=0.175]{comparar_NaCl.eps}
     \end{figure}
	\end{block}
	\begin{block}{KCl}
    \begin{figure}
     \includegraphics[scale=0.175]{comparar_KCl.eps}
     \end{figure}
	\end{block}
\end{column}
\begin{column}{4cm}
	\begin{block}{NaBr}
    \begin{figure}
     \includegraphics[scale=0.175]{comparar_NaBr.eps}
     \end{figure}
	\end{block}
	\begin{block}{KBr}
    \begin{figure}
     \includegraphics[scale=0.175]{comparar_KBr.eps}
     \end{figure}
	\end{block}
\end{column}
\begin{column}{4cm}
	\begin{block}{NaI}
    \begin{figure}
	    \includegraphics[scale=0.175]{comparar_NaI.eps}
     \end{figure}
	\end{block}
	\begin{block}{KI}
    \begin{figure}
     \includegraphics[scale=0.175]{comparar_KI.eps}
     \end{figure}
	\end{block}
\end{column}
\end{columns}
\end{frame}



\begin{frame}{Valores ótimos de q}
	Resultados do ajuste do parâmetro $q$ da tensão superficial da proposta de generalização ($u^{im}_q$)
	da equação de Onsager-Samaras (1934).
	\begin{table}
		\caption{Parâmetro $q$ ajustado para dados experimentais de tensão superficial de soluções aquosas
		de eletrólitos concentrados}
		\begin{tabular}{l|lll}
		&            q   &       $r_{+}$ &    $r_{-}$  \\\hline
{NaCl}    	&         2.15    &     1.02       &    1.81    \\
{NaBr}    	&         2.65    &     1.02       &    1.96    \\
{NaI }    	&          1.15   &      1.02      &     2.20   \\ 
{KCl }    	&          2.15   &      1.38      &    1.81    \\
{KBr }    	&          2.00   &      1.38      &    1.96    \\
{KI  }    	&           1.15  &       1.38     &     2.20   
		\end{tabular}
	\end{table}

\end{frame}

\begin{frame}{Teoria de Condensação de Contraíons - Manning-1969}

%A formulação da teoria de condensação de contraíons desenvolvida
%por Manning (1969)~\cite{Manning1969} para a solução de 
%eletrólitos permanece relevante na explicação do 
%comportamento de suas propriedades coligativas. Manning 
%tratou este problema com as seguintes hipóteses:

\begin{enumerate}
	\item A cadeia polimérica é substituída por uma 
		linha infinita uniformemente carregada;
	\item A interação entre dois ou mais poli-íons 
		são negligenciadas; 
	\item O solvente é considerado como meio contínuo
		e a constante dielétria é considerada 
		como a mesma do solvente puro para a 
		mesma temperatura; 
	\item Numa solução diluída um número suficiente de 
		contraíons condensará sobre o poli-íon 
		para reduzir a densidade de carga do polieletrólito 
		que o mantém acima do valor do parâmetro de 
		estabilidade $\xi$;
	\item Íons não condensados são tratados pela aproximação de 
		Debye-H{\"u}ckel.
	\end{enumerate}
\end{frame}
\begin{frame}{Polieletrólitos}
	\begin{figure}
		\includegraphics[width=6cm]{manning-diagrama.eps}
	\end{figure}

\begin{equation}
 \psi(u) = 2\frac{\beta}{\epsilon}\int^{+\infty}_{1} \frac{\exp(-\kappa u t)}{\sqrt{t^2-1}}dt\,,
\end{equation}
	\begin{block}{Representação por função de Bessel modificada de segunda espécie e ordem zero.}
	\begin{equation}
 \psi(u) = 2\frac{\beta}{\epsilon}  K_0(u).
	\label{eq:manning_sol}
\end{equation}
	\end{block}
\end{frame}
\begin{frame}{Proposta}
        %
\begin{equation}
 K_{0,q}(u) =  \int_{1}^{\infty}\frac{\exp_q(-u t)}{\sqrt{t^2-1}}dt.
\end{equation}

\begin{equation}
 \label{eq:cylq}
 \psi_q(u) =  2\frac{\beta}{\epsilon}  K_{0,q}(\kappa u).
\end{equation}
\begin{figure}
\centering
	{\includegraphics[width=4.5cm]{cylindrical}}
	\qquad
	{\includegraphics[width=4.5cm]{cylindrical_log_linear}}
%\caption{
%	Potencial adimensionalizado como função da distância para a 
%	geometria cilíndrica.
%	Linha sólida vermelha: caso ordinário ($q=1$);
%	linha pontilhada preta: $q<1$; linha tracejada preta: $q>1$.
% (a) Escala linear-linear (b) Escala log-linear.
%}
 \label{fig:cylindrical_q}
\end{figure}



\end{frame}

%\section{Interfacial Tension in electrolyte solutions}
\section{Conclusão}

\begin{frame}

	\begin{itemize}
\item LR - Integração numérica de equação diferencial de Poisson-Boltzmann implementado em R. 
      Demonstrando a limitação da solução pelos métodos convencionais sistemas 
      com alta concentração. Embora não tenha sido implementado, a integração 
      da equação diferencial por abordagens de métodos de Monte-Carlo ou correlatos 
      pode ser um caminho para esta dificuldade Holst 1995; 
\item LR - Embora tenha sido explorado de maneira eficaz as propriedades de tensão superficial 
      e coeficiente de atividade iônico médio para soluções de eletrólitos fortes existe 
      uma lacuna dos efeitos combinados das energias de solvatação e o quanto apenas o termo de Born 
      seria suficiente para representar o restante da escala de molalidade; 
\item LR - A elaborada conjectura apresentada entre os comportamentos monotônicos para 
      concentrações moderadas de eletrólitos univalentes  e a condição de depleção de
      íons é importante que seja mais aprofundada com outras medições da camada 
      de interface; 
	\end{itemize}
\end{frame}

\begin{frame}
	\begin{itemize}
\item LR - O comportamento da força da imagem do íon oscilatória ao longo de R 
      com decaimento da cauda seguindo aparentemente lei de potência sugere a 
      necessidade de proprosição de modelos que estão dentro escopo da Mecânica 
      estatistica não extensiva, como explicado no capítulo de aspectos matemático 
      deste documento de qualificação; 
\item LR - O texto da qualificação deixa uma lacuna de desenvolvimento ao propor 
      as relações generalizado para a camada dupla em geometria plana e o problema 
      de condensação de contraíons sem demonstrar as aplicações. Umas das principais
      vantagens das funções desenvolvidas está na não limitação do potencial da 
      superfície de contorno, que para a condição de solução da equação de Poisson
      Boltzmann é necessária. 
\item Para a conclusão da tese o doutorando finalizará a descrição de sistemas simples 
      que auxiliem no bom entendimento da tese. 
\item No documento final serão incluídos os algorítimos implementado nos softwares R, Fortran
      e interfaces construídas em Shiny. 
\end{itemize}

\end{frame}

\begin{frame}
	\begin{itemize}
\item LR - O texto da qualificação deixa uma lacuna de desenvolvimento ao propor 
      as relações generalizado para a camada dupla em geometria plana e o problema 
      de condensação de contraíons sem demonstrar as aplicações. Umas das principais
      vantagens das funções desenvolvidas está na não limitação do potencial da 
      superfície de contorno, que para a condição de solução da equação de Poisson
      Boltzmann é necessária. 
\item Para a conclusão da tese o doutorando finalizará a descrição de sistemas simples 
      que auxiliem no bom entendimento da tese. 
\item No documento final serão incluídos os algorítimos implementado nos softwares R, Fortran
      e interfaces construídas em Shiny. 
	\end{itemize}
\end{frame}


\end{document}
